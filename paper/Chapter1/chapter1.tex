\chapter{Understanding datasets}

\section{Source of the data}

\subsection{Barkley Earth}

Berkeley Earth is independent, non-profit organization founded by Richard and Elizabeth Muller in 2010, focused on environmental data sience and analysis. 
One of the main purposes of the project is to raise awareness about global warming. 
Berkeley Earth provides open source air polution and global temperature data \cite{berkeleyearthdata}. 
Their datasets cover 250 years of Earth temperature.

Berkeley Earth publishes their data in two type of formats
\begin{itemize}
    \item Time Series Data
    \item Gridded Data
\end{itemize}

{\it Time Series Data} summarises tabular format of yearly, monthly and daily anomalies of land's temperature. 
Below there is an example of such representation
\begin{verbatim}
%               Monthly    Annual    Five-year  Ten-year   Twenty-year
% Year, Month,  Anomaly,   Anomaly,  Anomaly,   Anomaly,   Anomaly
  ...
  1987     1     0.587     0.255     0.270      0.339      0.344
  1987     2     1.008     0.249     0.274      0.344      0.345
  1987     3     0.017     0.300     0.281      0.353      0.346
  1987     4     0.300     0.323     0.287      0.354      0.347
  ...
\end{verbatim}

{\it Gridded Data} adds additional dimension to a dataset. A grid is represended in two forms, as 1º x 1º {\it latitude-longitude} representation or {\it Equal Area} format that divides Earth into 15984 equaled cells.
{\it Gridded Data} is described by NetCDF format, that is described in Appendix \ref{app:netcdf}

\subsection{Our World in Data}

\section{Understanding data format}

\section{Datasets Insights}

\section{Summary}